\section{Training}%
\label{sec:train}

\begin{frame}{Loss-Funktion}
  \begin{itemize}
  \item Dient zur Berechnung des Fehlers während dem Training
  \item Trainingsfehler soll minimiert werden
  \item \(\Rightarrow\) wir suchen den Punkt, an dem die Ableitung der
    Loss-Funktion \(0\) wird, der Fehler also nicht mehr abnimmt
  \item Es gibt eine Vielzahl an Loss-Funktionen, wir betrachten hier
    die \enquote{\textbf{Mean Squared Error (MSE)}}:
  \end{itemize}

  \vspace{.6cm}

  \begin{minipage}{.4\linewidth}
    \[C(w, b) = \frac{1}{2m} \sum_{x=1}^{m} (y(x) - \hat{y}(x))^2\]
  \end{minipage}\hfill%
  \begin{minipage}{.55\linewidth}
    \uncover<2->{
      \begin{tabular}{ll}
        \toprule
        \(C(w, b)\) & Cost in Abhängigkeit von \(w\) und \(b\)\\
        \(m\) & Anzahl der Trainingsinstanzen\\
        \(y(x)\) & Gewünschter Output wenn \(x\) Input ist\\
        \(\hat{y}(x)\) & Tatsächlicher Output des Netzwerkes\\
        \bottomrule
      \end{tabular}
    }
  \end{minipage}
\end{frame}

\subsection{Gradient Descent}%
\label{sec:graddesc}

\begin{frame}{Gradient Descent}
  \begin{itemize}
  \item Methode um die Weights \(w\) und Biases \(b\) zu optimieren
  \item Vorgehen: \pause
    \begin{enumerate}
    \item Finde die Änderungsrate des Fehlers in Abhängigkeit von den Weights und Biases (\(\partial C / \partial w; \partial C / \partial b\))\pause
    \item Multipliziere die Änderungsrate mit der Lernrate \(\eta\)\pause
    \item Ziehe das Produkt aus Änderungsrate und Lernrate von den aktuellen Parametern ab\pause
    \item Aktualisiere die alten Parameter durch das Ergebnis des letzten Schrittes
    \end{enumerate}
  \end{itemize}

  \begin{align*}
    w_{k+1} &= w_{k} - \eta \frac{\partial C}{\partial w_{k}}\\[1em]
    b_{k+1} &= b_{k} - \eta \frac{\partial C}{\partial b_{k}}\\
  \end{align*}
\end{frame}

\subsection{Backpropagation}%
\label{sec:backprop}

\begin{frame}{Backpropagation --- Einleitung}
  \only<-2>{%
    \begin{alertblock}{Problem}
      Wie finde ich die Änderungsraten \(\frac{\partial C}{\partial w}; \frac{\partial C}{\partial b}\), die ich für Gradient Descent benötige?
    \end{alertblock}
  }

  \uncover<2->{%
    \textbf{Beispiel:}

    Gegeben:
    \[f(a, b, c, d) = {({(a + b)} \cdot (c + d))}^{2}\]
    Gesucht:
    \[\frac{\partial f}{\partial a}\]

    \only<3->{%
      \textbf{Analytische Lösung}

      Umformung zu Summanden:
      \begin{align*}
        f &= (a + b)^{2} \cdot (c + d)^{2}\\
        f &= (a^{2} + 2ab + b^{2}) \cdot (c + d)^{2}\\
      \end{align*}
      Ableitung:
      \[\frac{\partial f}{\partial a} = (2a + 2b) \cdot (c + d)^{2}\]
    }
  }

  
\end{frame}

\begin{frame}{Backpropagation --- Idee}
  
  \begin{itemize}
  \item \textbf{Problem:}
    \begin{itemize}
    \item Analytische Verfahren kommen bei komplizierten Funktionen an ihre Grenzen
    \item Sind schwer zu implementieren
    \item Sind ineffizient
    \end{itemize}\pause
  \item \textbf{Idee:} Divide and conquer; Problem in kleinere, handhabbare Probleme zerlegen \parencite{rumelhart1986learning, stanford19}
  \end{itemize}\pause

  \begin{align*}
    \uncover<3->{f(a, b, c, d) &= {({(a + b)} \cdot (c + d))}^{2}\\}
    \uncover<4->{g &= a + b\\}
    \uncover<5->{h &= c + d\\}
    \uncover<6->{i &= g \cdot h\\}
    \uncover<7->{f &= i^{2}}
  \end{align*}
\end{frame}

\begin{frame}{Backpropagation}
  \begin{columns}
    \begin{column}{.4\textwidth}
      \begin{align*}
        f(a, b, c, d) &= {({(a + b)} \cdot (c + d))}^{2}\\
        g &= a + b\\
        h &= c + d\\
        i &= g \cdot h\\
        f &= i^{2}
      \end{align*}
    \end{column}
    \begin{column}{.6\textwidth}
      \begin{tikzpicture}[every node/.style={circle, minimum width=.8cm}]
        \node (a) at (0, 7) {\(a\)};
        \node (b) at (0, 5) {\(b\)};
        \node (c) at (0, 3) {\(c\)};
        \node (d) at (0, 1) {\(d\)};
        \node[draw] (g) at (2, 6) {\(+\)};
        \node[draw] (h) at (2, 2) {\(+\)};
        \node[draw] (i) at (4, 4) {\(\cdot\)};
        \node[draw] (j) at (6, 4) {\(i^{2}\)};
        \draw (a) -- (g);
        \draw (b) -- (g);
        \draw (c) -- (h);
        \draw (d) -- (h);
        \draw (g) -- (i) node[midway, above] {\(g\)};
        \draw (h) -- (i) node[midway, above] {\(h\)};
        \draw (i) -- (j) node[midway, above] {\(i\)};
        \draw (j) -- (7, 4) node[midway, above] {\(f\)};
      \end{tikzpicture}
    \end{column}
  \end{columns}
\end{frame}

\begin{frame}{Backpropagation}

  \vspace{-.2cm}

  \begin{center}
    \begin{tikzpicture}[every node/.style={circle, minimum width=.8cm}]
      \node (a) at (0, 6) {\(a\)};
      \node (b) at (0, 4) {\(b\)};
      \node (c) at (0, 3) {\(c\)};
      \node (d) at (0, 1) {\(d\)};
      \node[draw] (g) at (4, 5) {\(+\)};
      \node[draw] (h) at (4, 2) {\(+\)};
      \node[draw] (i) at (6, 3.5) {\(\cdot\)};
      \node[draw] (j) at (9, 3.5) {\(i^{2}\)};
      \only<-2>{\draw (a) -- (g);}
      \only<3->{\draw (a) -- (g) node[midway,above]{\(\frac{\partial g}{\partial a}\)};}
      \draw (b) -- (g);
      \draw (c) -- (h);
      \draw (d) -- (h);
      \only<-2>{\draw (g) -- (i) node[midway, above] {\(g\)};}
      \only<3->{\draw (g) -- (i) node[midway, above] {\(g\)} node[midway, below] {\(\frac{\partial gh}{\partial g}\)};}
      \draw (h) -- (i);
      \only<-2>{\draw (i) -- (j) node[midway, above] {\(i\)};}
      \only<3->{\draw (i) -- (j) node[midway, above] {\(i\)} node[midway, below]{\(\frac{\partial i}{\partial gh}\)};}
      \only<-2>{\draw (j) -- (11, 3.5) node[midway, above] {\(f\)};}
      \only<3->{\draw (j) -- (11, 3.5) node[midway, above] {\(f\)} node[midway, below]{\(\frac{\partial f}{\partial i}\)};}
    \end{tikzpicture}
  \end{center}

  \vspace{-.4cm}

  \begin{alertblock}{Gesucht: \(\frac{\partial f}{\partial a}\) \parencite{stanford19, colah15}}
    \uncover<2->{\[\frac{\partial f}{\partial a} = \frac{\partial f}{\partial i} \cdot \frac{\partial i}{\partial gh} \cdot \frac{\partial gh}{\partial g} \cdot \frac{\partial g}{\partial a}\]}
  \end{alertblock}
\end{frame}

\begin{frame}{Beispiel}
  \begin{center}
    \begin{tikzpicture}[every node/.style={circle, minimum width=.8cm}]
      \node (a) at (0, 6) {\(1\)};
      \node (b) at (0, 4) {\(2\)};
      \node (c) at (0, 3) {\(3\)};
      \node (d) at (0, 1) {\(4\)};
      \node[draw] (g) at (4, 5) {\(+\)};
      \node[draw] (h) at (4, 2) {\(+\)};
      \node[draw] (i) at (6, 3.5) {\(\cdot\)};
      \node[draw] (j) at (9, 3.5) {\(i^{2}\)};
      \only<-8>{\draw (a) -- (g);}
      \draw (b) -- (g);
      \draw (c) -- (h);
      \draw (d) -- (h);

      \only<1>{\draw (g) -- (i);}
      \only<2-7>{\draw (g) -- (i) node[midway, above] {\(3\)};}
      \only<-2>{\draw (h) -- (i);}
      \uncover<3->{\draw (h) -- (i) node[midway, above] {\(7\)};}
      \only<-3>{\draw (i) -- (j);}
      \only<4-7>{\draw (i) -- (j) node[midway, above] {\(21\)};}
      \only<-4>{\draw (j) -- (11, 3.5);}
      \only<5>{\draw (j) -- (11, 3.5) node[midway, above] {\(441\)};}
      \uncover<6->{\draw (j) -- (11, 3.5) node[midway, above] {\(441\)} node[midway, below]{\(\frac{\partial f}{\partial i}\)};}
      \uncover<7->{\draw (i) -- (j) node[midway, above] {\(21\)} node[midway, below]{\(\frac{\partial i}{\partial gh}\)};}
      \uncover<8->{\draw (g) -- (i) node[midway, above] {\(3\)} node[midway, below] {\(\frac{\partial gh}{\partial g}\)};}
      \only<9->{\draw (a) -- (g) node[midway,above]{\(\frac{\partial g}{\partial a}\)};}
    \end{tikzpicture}

    \uncover<10->{\[\frac{\partial f}{\partial i} = 2 \cdot 21 = 42; \quad \frac{\partial i}{\partial gh} = 1; \quad \frac{\partial gh}{g} = h = 7; \quad \frac{\partial g}{\partial a} = 1 \quad \Rightarrow 7 \cdot 42 = 294\]}

  \end{center}
\end{frame}

\begin{frame}{Überprüfung mit analytischer Lösung}
  Gegeben:
  \[f(a, b, c, d) = {({(a + b)} \cdot (c + d))}^{2},\quad a=1, b=2, c=3, d=4\]
  Gesucht:
  \[\frac{\partial f}{\partial a}\]
  Gefunden:
  \[\frac{\partial f}{\partial a} = (2a + 2b) \cdot (c + d)^{2}\]\pause
  Einsetzen:
  \[\frac{\partial f}{\partial a} = (2 \cdot 1 + 2 \cdot 2) \cdot (3 + 4)^{2} = 294\]
\end{frame}

%%% Local Variables:
%%% mode: latex
%%% TeX-master: "../präsentation"
%%% End:
