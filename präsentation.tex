\documentclass[aspectratio=169, 10pt]{beamer}

\usetheme{metropolis}
\usetikzlibrary{shapes, snakes, spy, positioning}

\setbeamercolor{background canvas}{bg=white}

\metroset{
  numbering=fraction,
  progressbar=frametitle,
  block=fill,
}

\usefonttheme[onlymath]{serif}

\usepackage[ngerman]{babel}
\usepackage[style=iso-numeric]{biblatex}
\usepackage{csquotes}
\usepackage{mathtools, graphicx, booktabs, pgfplots, xcolor}
\usepackage[font=footnotesize]{caption}
\usepackage{listings}
\usepackage{hyperref}

\definecolor{mDarkBrown}{HTML}{604c38}
\definecolor{mDarkTeal}{HTML}{23373b}
\definecolor{mLightBrown}{HTML}{EB811B}
\definecolor{mLightGreen}{HTML}{14B03D}
\definecolor{mBackground}{HTML}{FFFFFF}
\definecolor{myblue}{HTML}{6699cc}
\definecolor{myorange}{HTML}{eb801a}

\lstset{%
  language=Python,
  basicstyle=\small\ttfamily,
  keywordstyle=\color{mLightBrown}\bfseries,
  commentstyle=\color{mLightGreen},
  stringstyle=\color{mLightGreen},
  backgroundcolor=\color{mBackground},
  numbers=none,
  numberstyle=\tiny\ttfamily,
  stepnumber=2,
  showspaces=false,
  showstringspaces=false,
  showtabs=false,
  frame=none,
  framerule=1pt,
  tabsize=2,
  rulesep=5em,
  captionpos=b,
  breaklines=true,
  breakatwhitespace=false,
  framexleftmargin=0em,
  framexrightmargin=0em,
  xleftmargin=0em,
  xrightmargin=0em,
  aboveskip=1em,
  belowskip=1em,
  morekeywords={usetheme,institute,maketitle,@metropolis@titleformat,%
                plain,setbeamercolor,metroset,setsansfont,setmonofont},
}


\addbibresource{bibliography.bib}
\graphicspath{{./Grafiken/}}
\captionsetup{belowskip=0pt}

\newcommand{\dnn}[1]{
  \def\layersep{2.5cm}
  \def\list{#1}
  \begin{tikzpicture}[->, >=stealth, node distance=\layersep]
    \tikzstyle{every pin edge}=[<-,shorten <=1pt]
    \tikzstyle{neuron}=[circle,draw,minimum size=17pt,inner sep=0pt]
    \pgfmathsetmacro{\numl}{dim({\list})}
    \pgfmathsetmacro{\numh}{\numl-1}
    \pgfmathsetmacro{\numi}{{\list}[0]}
    \pgfmathsetmacro{\numo}{{\list}[\numl-1]}

    % Draw the input layer
    \foreach \y in {1,...,\numi}
    \node[neuron, pin=left:\(x_{\y}\)] (I-\y) at (0,-\y) {};

    % Draw the hidden layers
    \foreach \y [count=\i, evaluate=\i as \break using \i + 2] in \list {
      \pgfmathsetmacro{\counti}{{\list}[\i]}
      \foreach \x in {1,...,\counti} {
        \path[yshift=(\counti cm - \numi cm)/2] node[neuron] (H\i-\x) at (\i * \layersep, -\x cm) {};
      }
      \ifnum \numl=\break
      \breakforeach
      \fi
    }

    % Draw the output layer nodes
    \foreach \y in {1,...,\numo}
    \path[yshift=(\numo cm - \numi cm)/2] node[neuron,pin={[pin edge={->}]right:\(y_{\y}\)}] (O-\y)  at (\numh * \layersep, -\y cm) {};

    % Connect Input layer and first hidden layer
    \pgfmathsetmacro{\lhonenum}{{\list}[1]}
    \foreach \source in {1,...,\numi}
    \foreach \dest in {1,...,\lhonenum}
    \path (I-\source) edge (H1-\dest);

    \foreach \y [count=\i, evaluate=\i as \break using \i + 3] in \list {
      \pgfmathsetmacro{\curr}{int({\list}[\i])}
      \pgfmathsetmacro{\next}{int({\list}[\i+1])}
      \pgfmathsetmacro{\j}{int(\i+1)}
      \foreach \source in {1,...,\curr}
      \foreach \dest in {1,...,\next}
      \path (H\i-\source) edge (H\j-\dest);
      \ifnum \numl=\break
      \breakforeach
      \fi
    }

    % Connect last hidden layer with output layer
    \pgfmathsetmacro{\lmone}{{\list}[\numl-2]}
    \pgfmathsetmacro{\lmonenum}{int(dim({\list})-2)}
    \foreach \source in {1,...,\lmone}
    \foreach \dest in {1,...,\numo}
    \path (H\lmonenum-\source) edge (O-\dest);
    \node[node distance=1.9cm, above of=H2-1] (hidden) {Hidden Layers};
    \node[node distance=2*\layersep, left of=hidden] {Input Layer};
    \node[node distance=2*\layersep, right of=hidden] {Output Layer};
    \draw[-, decorate, decoration=brace, thick] (2.1cm, .5cm) -- (7.8cm, .5cm);
  \end{tikzpicture}
}

%%% Local Variables:
%%% mode: latex
%%% TeX-master: "../präsentation"
%%% End:


\title{Algorithmik zur Optimierung in neuronalen Netzwerken}
\subtitle{Gradient Descent und Backpropagation}
\institute{Hochschule Esslingen --- University of Applied Sciences}
\author{Tim Hilt}
\date{19. Mai 2020}

\begin{document}

\begin{frame}
  \maketitle
\end{frame}

\begin{frame}{Gliederung}
  \tableofcontents
\end{frame}

\section{Supervised Learning}%
\label{sec:supervised}

\begin{frame}{Machine Learning Workflow}
  \begin{figure}[!htbp]
    \centering
    \includegraphics[width=10cm]{ml_workflow}
    \caption{Machine Learning Workflow \parencite{geron2019hands}}%
    \label{fig:workflow}
  \end{figure}
\end{frame}

\begin{frame}{Supervised Learning}
  \begin{figure}[!htbp]
    \centering
    \includegraphics[width=8cm]{data}
    \caption{Struktur der Daten bei Supervised Learning \parencite{geron2019hands}}%
    \label{fig:data}
  \end{figure}

  \only<2->{
    \small
    \begin{block}{Definition Supervised Learning}
      \enquote{In supervised learning, the dataset is the collection of labeled examples
      \({\{(\mathbf{x}_i, y_i)\}}_{i=1}^N\). Each element \(\mathbf{x}_i\) among
      \(N\) is called a feature vector. A feature vector is a vector in which each
      dimension \(j = 1, \ldots , D\) contains a value that describes the example
      somehow [\ldots]. The goal of a supervised learning algorithm is to use the
      dataset to produce a model, that takes a feature vector \(\mathbf{x}\) as
      input and outputs information that allow deducing the label \(\hat{y}\) for
      this feature vector.} \parencite{burkov2019hundred}
    \end{block}
  }
\end{frame}

\begin{frame}{Beispiel: Datensatz für Supervised Learning}
  \begin{minipage}{.6\textwidth}
    \centering
    \includegraphics[width=\textwidth]{mnist}
  \end{minipage}\hfill%
  \begin{minipage}{.4\textwidth}
    \begin{itemize}
    \item Insgesamt 70000 Bilder
    \item Bildgröße: 28 \(\times\) 28 Pixel
    \item Abgebildet: Handgeschriebene Ziffern von 0 bis 9
    \item Quelle: Yann LeCun et al \parencite{lecun1998gradient}
    \end{itemize}
  \end{minipage}
\end{frame}

\begin{frame}{Beispiel: Datensatz für Supervised Learning}
  \begin{minipage}{.6\textwidth}
    \centering \includegraphics[width=\textwidth]{fashion_mnist}
  \end{minipage}\hfill%
  \begin{minipage}{.4\textwidth}
    \begin{itemize}
    \item Insgesamt 70000 Bilder
    \item Bildgröße: 28 \(\times\) 28 Pixel
    \item Abgebildet: Kleidungsstücke 
    \item Quelle: Zalando Research \parencite{xiao2017fashion}
    \end{itemize}

    \vspace{.5cm}

    \tiny

    \centering
    \begin{tabular}{ll}
      \toprule
      \textbf{Label} & 	\textbf{Description}\\
      \midrule
      0 & 	T-shirt/top\\
      1 & 	Trouser\\
      2 & 	Pullover\\
      3 & 	Dress\\
      4 & 	Coat\\
      5 & 	Sandal\\
      6 & 	Shirt\\
      7 & 	Sneaker\\
      8 & 	Bag\\
      9 & 	Ankle boot\\
      \bottomrule
    \end{tabular}

  \end{minipage}

\end{frame}

%%% Local Variables:
%%% mode: latex
%%% TeX-master: "../präsentation"
%%% End:

\section{Künstliche Neuronale Netze}%
\label{sec:ann}

\begin{frame}{Künstliches Neuron}
  \begin{minipage}{.65\textwidth}
    \begin{tikzpicture}[->, >=stealth, thick, scale=.8]
      \node [circle split,draw,rotate=90] (z) at (0, 0) {\rotatebox{-90}{$\displaystyle\sum$} \nodepart{lower} \rotatebox{-90}{$\sigma(z)$}};
      \foreach \i in {5,...,1} {
        \node (x-\i) at (-7, 4.5+1.5*-\i) {\(x_{\i}\)};
        \path (x-\i) edge node[above] {\(w_{\i}\)} (z);
      }
      \node (y) at (4, 0) {\(a\)};
      \path (z) edge (y);
      \node (b) at (0, 2.5) {\(b\)};
      \draw (b) -- (z);
    \end{tikzpicture}
  \end{minipage}%
  \begin{minipage}{.35\textwidth}
    \uncover<2->{
      \begin{flushright}
        \begin{tabular}{cl}
          \toprule
          \(\mathbf{x}\)  & Inputvektor\\
          \(\mathbf{w}\)  & Gewichtsvektor\\
          \(b\)  & Bias\\
          \(z\)  & Zwischensumme \(\left(\sum\right)\)\\
          \(\sigma(x)\)  & Aktivierungsfunktion\\
          \(a\)  & Aktivierung/ Output\\
          \bottomrule
        \end{tabular}
      \end{flushright}
    }
  \end{minipage}

  \only<3>{
    \[z = \sum_{i}x_{i}w_i + b = \mathbf{xw} + b\]

    {\centering\(\Rightarrow z\) wird für spätere Parameteroptimierung benötigt\par}
  }
\end{frame}

\begin{frame}{Aktivierungsfunktion \(\sigma(x)\)}
  \begin{minipage}{.45\textwidth}
    \(\Rightarrow\) Es gibt eine Vielzahl verschiedener Aktivierungsfunktionen für unterschiedliche Problemstellungen, für uns soll jedoch lediglich die \textbf{Sigmoid-Funktion} relevant sein:

    \vspace{1cm}

    \[\sigma(x) = \frac{1}{1 + e^{-x}}\]
  \end{minipage}\hfill%
  \begin{minipage}{.5\textwidth}
    \only<2>{
      \begin{tikzpicture}
        \begin{axis}[width=\textwidth, mlineplot, samples=50, ymin=-.1, ymax=1.1, domain=-10:10, title=Sigmoid-Funktion, xlabel=\(x\), ylabel=\(\sigma(x)\)]
          \addplot{1/(1+exp(-x))};
        \end{axis}
      \end{tikzpicture}
    }
  \end{minipage}

\end{frame}

\begin{frame}{Architektur eines Neuronalen Netzwerks}
  \def\layersep{3.5cm}
  \begin{center}
    \alt<2>{
      \begin{tikzpicture}[->, >=stealth, node distance=\layersep]
        \tikzstyle{every pin edge}=[<-,shorten <=1pt]
        \tikzstyle{neuron}=[circle,draw,minimum size=17pt,inner sep=0pt]
        
        % Draw the input layer nodes
        \foreach \y in {1,...,3}
        \node[neuron, pin=left:\(x_{\y}\)] (I-\y) at (0,-\y) {};

        % Draw the nodes for the second hidden layer
        \foreach \y in {1,...,5}
        \path[yshift=(5cm - 3cm)/2] node[neuron] (H1-\y) at (\layersep,-\y cm) {};

        % Draw the output layer nodes
        \foreach \y in {1,...,2}
        \path[yshift=(2cm - 3cm)/2] node[neuron,pin={[pin edge={->}]right:\(y_{\y}\)}] (O-\y) at (2*\layersep,-\y cm) {};

        \foreach \source in {1,...,3}
        \foreach \dest in {1,...,5}
        \path (I-\source) edge (H1-\dest);

        \foreach \source in {1,...,5}
        \foreach \dest in {1,...,2}
        \path (H1-\source) edge (O-\dest);

        \node[above of=H1-1, node distance=1cm] (hidden) {Hidden Layer};
        \node[right of=hidden] (output) {Output Layer};
        \node[left of=hidden] (input) {Input Layer};
      }{
        \begin{tikzpicture}[->, >=stealth]
          \tikzstyle{every pin edge}=[<-,shorten <=1pt]
          \tikzstyle{neuron}=[circle,draw,minimum size=17pt,inner sep=0pt]
          
          % Draw the input layer nodes
          \foreach \y in {1,...,3}
          \node[neuron, pin=left:] (I-\y) at (0,-\y) {};

          % Draw the nodes for the second hidden layer
          \foreach \y in {1,...,5}
          \path[yshift=(5cm - 3cm)/2] node[neuron] (H1-\y) at (\layersep,-\y cm) {};

          % Draw the output layer nodes
          \foreach \y in {1,...,2}
          \path[yshift=(2cm - 3cm)/2] node[neuron,pin={[pin edge={->}]right:}] (O-\y) at (2*\layersep,-\y cm) {};

          \foreach \source in {1,...,3}
          \foreach \dest in {1,...,5}
          \path (I-\source) edge (H1-\dest);

          \foreach \source in {1,...,5}
          \foreach \dest in {1,...,2}
          \path (H1-\source) edge (O-\dest);
        }
      \end{tikzpicture}
    \end{center}
  \end{frame}

  \begin{frame}{Deep Neural Network}
    \begin{center}
      \dnn{5, 6, 5, 7, 3}
    \end{center}
  \end{frame}

  \begin{frame}{Target-Architektur zur Klassifikation von MNIST}
    \def\layersep{2.5cm}
    \begin{center}
      \begin{tikzpicture}[->, >=stealth, node distance=\layersep]
        \tikzstyle{every pin edge}=[<-,shorten <=1pt]
        \tikzstyle{neuron}=[circle,draw,minimum size=17pt,inner sep=0pt]
        
        \node[neuron, pin=left:\(x_{1}\)] (I-1) at (0,-1) {}; \node[neuron,
        pin=left:\(x_{2}\)] (I-2) at (0,-2) {}; \node[neuron,
        pin=left:\(x_{3}\)] (I-3) at (0,-3) {}; \node at (0, -4) {\(\vdots\)};
        \node[neuron, pin=left:\(x_{784}\)] (I-4) at (0,-5) {};

        \path[yshift=(6cm - 5cm)/2] node[neuron] (H1-1) at (\layersep,-1 cm) {};
        \path[yshift=(6cm - 5cm)/2] node[neuron] (H1-2) at (\layersep,-2 cm) {};
        \path[yshift=(6cm - 5cm)/2] node[neuron] (H1-3) at (\layersep,-3 cm) {};
        \path[yshift=(6cm - 5cm)/2] node[neuron] (H1-4) at (\layersep,-4 cm) {};
        \path[yshift=(6cm - 5cm)/2] node at (\layersep, -4.7cm) {$\vdots$};
        \path[yshift=(6cm - 5cm)/2] node[neuron] (H1-5) at (\layersep,-6 cm) {};

        \path node[neuron] (H2-1) at (2*\layersep,-1 cm) {}; \path node[neuron]
        (H2-2) at (2*\layersep,-2 cm) {}; \path node[neuron] (H2-3) at
        (2*\layersep,-3 cm) {}; \path node at (2*\layersep, -3.7cm) {$\vdots$};
        \path node[neuron] (H2-4) at (2*\layersep,-5 cm) {};

        \path[yshift=(2cm - 3cm)/2] node[neuron,pin={[pin
          edge={->}]right:\(y_{1}\)}] (O-1) at (3*\layersep,-1 cm) {};
        \path[yshift=(2cm - 3cm)/2] node[neuron,pin={[pin
          edge={->}]right:\(y_{2}\)}] (O-2) at (3*\layersep,-2 cm) {};
        \path[yshift=(2cm - 3cm)/2] node at (3*\layersep, -3cm) {$\vdots$};
        \path[yshift=(2cm - 3cm)/2] node[neuron,pin={[pin
          edge={->}]right:\(y_{10}\)}] (O-3) at (3*\layersep,-4 cm) {};

        \foreach \source in {1,...,4} \foreach \dest in {1,...,5} \path
        (I-\source) edge (H1-\dest);

        \foreach \source in {1,...,5} \foreach \dest in {1,...,4} \path
        (H1-\source) edge (H2-\dest);

        \foreach \source in {1,...,4} \foreach \dest in {1,...,3} \path
        (H2-\source) edge (O-\dest);

        \foreach \y [count=\i] in {1,2,3,4,30} \node[node distance=14pt, above
        of=H1-\i] {\tiny\(H^1_{\y}\)};

        \foreach \y [count=\i] in {1,2,3,15} \node[node distance=14pt, above
        of=H2-\i] {\tiny\(H^2_{\y}\)};
      \end{tikzpicture}
    \end{center}
  \end{frame}

  %%% Local Variables:
  %%% mode: latex
  %%% TeX-master: "../präsentation"
  %%% End:

\section{Training}%
\label{sec:train}

\subsection{Loss-Funktion}%
\label{sec:loss}



\subsection{Gradient Descent}%
\label{sec:graddesc}



\subsection{Backpropagation}%
\label{sec:backprop}

\begin{frame}{Backpropagation}
  Es werden vier Gleichungen benötigt:

  Error im Output-Layer:

  Error einzelner Neuronen:

  \[\delta_j^L = \frac{\partial C}{\partial a_j^L} \sigma' \left(z_j^L\right)\]

  Vektorisiert:

  \[\delta^L = \nabla_aC \odot \sigma'(z^L)\]

  Aufgelöst, wenn MSE benutzt:

  \[\delta^L = (a^L - y) \odot \sigma'(z^L)\]
\end{frame}

\begin{frame}{Backpropagation}
  Error im Layer \(l\) hinsichtlich Error im nächsten Layer \(\delta^{l+1}\)

  \[\delta^l = \left(\left(w^{l+1}\right)^T\delta^{l+1}\right) \odot
    \sigma'\left(z^l\right)\]

  \begin{itemize}
  \item Rekursive Definition durch Verwendung von \(\delta^l\) in Abhängigkeit
    von \(\delta^{l+1}\)
  \item Wenn anfangs \(\delta^L\) in die Gleichung gegeben wird kann der Error
    rekursiv für jeden vorhergehenden Layer berechnet werden
  \end{itemize}
\end{frame}

\begin{frame}{Backpropagation}
  \begin{align*}
    \delta^L &= \nabla_aC \odot \sigma' \left(z^L\right)\\[1em]
    \delta^l &= \left(\left(w^{l+1}\right)^T \delta^{l+1}\right) \odot
               \sigma'\left(z^l\right)\\[1em]
    \frac{\partial C}{\partial b^l_j} &= \delta^l_j\\[1em]
    \frac{\partial C}{\partial w^l_{jk}} &= a^{l-1}_k\delta^l_j
  \end{align*}
\end{frame}

\begin{frame}{Beispiel}
  Pass
\end{frame}

%%% Local Variables:
%%% mode: latex
%%% TeX-master: "../präsentation"
%%% End:

\section{Umsetzung in Python (mit Keras)}

\begin{frame}[fragile]{Zuvor beschriebene Architektur}
  \begin{lstlisting}
    from tensorflow import keras
    ...
    model = keras.models.Sequential([
      keras.layers.Flatten(input_shape=(28, 28)),
      keras.layers.Dense(30, activation='sigmoid'),
      keras.layers.Dense(15, activation='sigmoid'),
      keras.layers.Dense(10, activation='sigmoid'),
    ])

    model.compile(loss='mse',
                  optimizer=keras.optimizers.SGD(learning_rate=.8),
                  metrics=['accuracy'])

    history = model.fit(X_train, y_train_cat, epochs=10,
                        validation_data=(X_valid, y_valid_cat))
  \end{lstlisting}
\end{frame}

\begin{frame}{Verlauf während des Trainings}
  \begin{center}
    \begin{tikzpicture}
      \begin{axis}[mlineplot,legend style={
          cells={anchor=east},
          legend pos=outer north east,
        }, ymin=0, ymax=1.1, xmin=0.8, xmax=10.2, xlabel=Epoch]
        \addplot table [x=history, y=loss, col sep=comma] {history1.csv};
        \addlegendentry{Loss}
        \addplot table [x=history, y=accuracy, col sep=comma] {history1.csv};
        \addlegendentry{Accuracy}
      \end{axis}
    \end{tikzpicture}
  \end{center}
\end{frame}

\begin{frame}[fragile]{Optimierte Architektur}
  \begin{itemize}
  \item Vorteil: Schnellere Konvergenz
  \item Verwendung von optimierter Cost-, Activation- und Gradient-Descent-Funktion
  \end{itemize}
  \begin{lstlisting}
    model = keras.models.Sequential([
      keras.layers.Flatten(input_shape=(28, 28)),
      keras.layers.Dense(30, activation='relu'),
      keras.layers.Dense(15, activation='relu'),
      keras.layers.Dense(10, activation='softmax'),
    ])

    model.compile(loss='sparse_categorical_crossentropy',
                  optimizer='Adam',
                  metrics=['accuracy'])
  \end{lstlisting}
\end{frame}

\begin{frame}{Verlauf während des Trainings}
  \begin{center}
    \begin{tikzpicture}
      \begin{axis}[mlineplot,legend style={
          cells={anchor=east},
          legend pos=outer north east,
        }, ymin=0, ymax=1.1, xmin=0.8, xmax=10.2, xlabel=Epoch]
        \addplot table [x=history, y=loss, col sep=comma] {history2.csv};
        \addlegendentry{Loss}
        \addplot table [x=history, y=accuracy, col sep=comma] {history2.csv};
        \addlegendentry{Accuracy}
      \end{axis}
    \end{tikzpicture}
  \end{center}
\end{frame}


%%% Local Variables:
%%% mode: latex
%%% TeX-master: "../präsentation"
%%% End:


\begin{frame}[standout]
  Fragen?
\end{frame}

\printbibliography[title=Literaturverzeichnis]

\end{document}